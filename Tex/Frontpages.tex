%---------------------------------------------------------------------------%
%->> 封面信息及生成
%---------------------------------------------------------------------------%
%-
%-> 中文封面信息
%-
\title{中文题目可能会特别长比如有这么这么这么长}%中文题目,下边需要手动分行
\titleone{~中文题目可能会特别长}% 论文中文题目第一行
\titletwo{~比如有这么这么这么长}% 论文中文题目第二行
\major{~管理科学与工程}%学科专业
\researchfield{~能源战略与政策系统分析}%研究方向
\author{~你的名字}% 论文作者
\advisor{~你导师的名字~教授}% 指导教师:姓名 专业技术职务

\chinesedate{二〇一九~年~五~月}% word模板上是5月
%-
%-> 英文封面信息
%-
\englishtitle{Chinese title could be very very very very very long}% 论文英文题目
\englishauthor{Your name}% 论文作者
\englishadvisor{Prof. Your supervisior}% 指导教师
\englishdegreetype{Management}% 学位类别:参考教务系统和word模板
\englishmajor{Management Science and Engineering}% 二级学科专业名称:参考教务系统和word模板
\englishdate{May, 2019}% 毕业日期:模板上是May
%-
%-> 生成封面
%-
\maketitle% 生成中文封面
\makeenglishtitle% 生成英文封面
%-
%-> 作者声明
%-
\makedeclaration% 生成声明页
%-
%-> 中文摘要
%-
\chapter{摘\qquad 要}\chaptermark{摘\qquad 要}% 下边写中文摘要

本文给出了中国石油大学(北京)博士学位论文的写作规范和排版格式要求。文中格式可作为编排博士学位论文的格式模板,供博士研究生参考使用。
摘要、Abstract、目录等前置部分用罗马数字单独编排页码,宋体,五号,居中。从正文开始,页码一律用阿拉伯数字连续编排,引言首页作为第1页。封一、封二和封底不编排页码。页码必须标注在每页页脚位置,Arial字体,五号,居中。
正文段落首行缩进2字;除章节标题、中文图名、中文表名及表内文字等有特殊说明之外,正文的中文汉字为宋体,英文字符和阿拉伯数字为Times New Roman,字号小四。行距:多倍行距 1.25,间距:段前、段后均为0行,选择网格对齐选项。
摘要部分说明:
“摘要”是摘要部分的标题,不可省略。
标题“摘要”选用模板中的样式所定义的“标题1”,居中;或者手动设置成字体:黑体,居中,字号:小三,1.5倍行距,段前为0,段后11磅。
论文摘要是学位论文的缩影,文字要简练、明确。内容要包括目的、方法、结果和结论。单位制一律换算成国际标准计量单位制,除特别情况外,数字一律用阿拉伯数码。文中不允许出现插图。重要的表格可以写入。
摘要正文选用模板中的样式所定义的“正文”,每段落首行缩进2个汉字;或者手动设置成每段落首行缩进2个汉字,字体:中文宋体,英文和阿拉伯数字Times New Roman,字号:小四,行距:多倍行距 1.25,间距:段前、段后均为0行,选择网格对齐选项。
摘要正文后,列出3-5个关键词。“关键词:”是关键词部分的引导,不可省略。关键词请尽量用《汉语主题词表》等词表提供的规范词。
关键词与摘要之间空一行。“关键词”黑体,小四,加粗;关键词词间用分号间隔,末尾不加标点,3-5个,宋体,小四。
中、英文摘要一般为800~1000字。

\keywords{写作规范;排版格式;博士学位论文}% 中文关键词
%-
%-> 英文摘要
%-
{
    \ctexset {
        chapter = {
            format = \linespread{1.25}\zihao{-3}\bfseries\centering,
            number = \arabic{chapter},
            aftername = \quad,
            beforeskip = {0pt},
            afterskip = {0pt},
            pagestyle = plain,
        }
    }
    \chapter*{The Format Criterion of Doctoral Dissertation}%在个括号里输入英文标题
    \chaptermark{ABSTRACT}
    {
        \linespread{1.25}
        {\ }\par
    }
    \ctexset {
        chapter = {
            break={\par\relax},
            format = \linespread{1.5}\zihao{-3}\bfseries\centering,
            number = \arabic{chapter},
            aftername = \quad,
            beforeskip = {0pt},
            afterskip = {11pt},
            pagestyle = plain,
        }
    }
    \chapter{ABSTRACT}
}% 下边写英文摘要

内容应与“中文摘要”对应。使用第三人称,最好采用现在时态编写。
“ABSTRACT”不可省略。标题“ABSTRACT”选用模板中的样式所定义的“标题1”,居中;或者手动设置成字体:Times New Roman,粗体,居中,字号:小三,多倍行距1.5倍行距,段前为0,段后11磅。
标题“ABSTRACT”上方是论文的英文题目,字体:Times New Roman,居中,字号:小三,行距:多倍行距 1.25,间距:段前、段后均为0行,选择网格对齐选项。
Abstract正文选用设置成每段落首行缩进2字,字体:Times New Roman,字号:小四,行距:多倍行距 1.25,间距:段前、段后均为0行,选择网格对齐选项。
Key words与Abstract之间空一行,“Key words”加粗。Key words与中文“关键词”一致。词间用分号间隔,末尾不加标点,3-5个,Times New Roman,小四。

\englishkeywords{Write Criterion; Typeset Format; Template}% 英文关键词

%-> 创新点
%-
\chapter{创\quad 新\quad 点}\chaptermark{创\quad 新\quad 点}% 创新点标题


\begin{enumerate}

\item 说是

\item 最多

\item 有四

\item 条创新点,一共1000字以内

\end{enumerate}
%创新点
%---------------------------------------------------------------------------%
