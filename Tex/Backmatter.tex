\chapter{致\quad 谢}\chaptermark{致\quad 谢}% syntax: \chapter[目录]{标题}\chaptermark{页眉}
学位论文中不得书写与论文工作无关的人和事,对导师的致谢要实事求是。
一同工作的同志对本研究所做的贡献应在论文中做明确的说明并表示谢意。
这部分内容不可省略。一般不超过400字。
书写格式说明:
标题“致谢”选用模板中的样式所定义的“标题1”,居中;或者手动设置成字体:黑体,居中,字号:小三,1.5倍行距,段后11磅,段前为0。
致谢正文选用模板中的样式所定义的“正文”,每段落首行缩进2字;或者手动设置成每段落首行缩进2字,字体:中文宋体,英文和阿拉伯数字Times New Roman,字号:小四,行距:多倍行距 1.25,间距:段前、段后均为0行,选择网格对齐选项。


\chapter{个人简历、在学期间发表的学术论文及研究成果}

姓名,性别,××年××月出生,××族,××省××市(县)人。××年××月毕业于××大学××学院(系)××专业,获学士学位;××年××月毕业于××大学××学院(系)××专业,获硕士学位;××年××月至××年××月在××公司工作,担任××工程师;××年进入中国石油大学(北京)××学院(系),攻读××专业博士学位研究生。

学术论文及研究成果仅列出博士生攻读学位期间以中国石油大学(北京)为第一署名单位在公开刊物(会议论文集)上发表或已被录用的与学位论文或所学专业有关的学术论文及研究成果,或是以中国石油大学(北京)为完成单位之一获得的奖励或以中国石油大学(北京)为专利权人申请的专利,并注明属于学位论文内容的部分(章节),所有作者及其顺序、所发表的刊物名称(包括主办单位、是否被SCI、EI检索期刊)、时间、卷(期)与页码。其它时间、不涉及中国石油大学(北京)或与学位论文和所学专业无关的论文不得列出,在职博士所做工作不涉及中国石油大学(北京)的成果一律不写。尚未刊载,但已经接到正式录用函的学术论文,在每一篇后加括号注明已被××××杂志录用。(不要写预计发表时间)
书写格式说明:
标题“攻读博士学位期间发表学术论文”选用模板中的样式所定义的“标题1”,居中;或者手动设置成字体:黑体,居中,字号:小三,1.5倍行距,段后11磅,段前为0。
在学研究成果正文选用模板中的样式所定义的“正文”,每段落首行缩进2字;或者手动设置成每段落首行缩进2字,字体:中文宋体,英文和阿拉伯数字Times New Roman,字号:小四,行距:多倍行距 1.25,间距:段前、段后均为0行,选择网格对齐选项。

例:
1	作者1, 作者2. 论文题名. 中国科学. 2004, 卷(期): 起始页码-终止页码. 主办单位: 中国科学院。SCI检索期刊,本文SCI检索号: 123DX。(本博士学位论文第一章)


\cleardoublepage[plain]% 让文档总是结束于偶数页,可根据需要设定页眉页脚样式,如 [noheaderstyle]

